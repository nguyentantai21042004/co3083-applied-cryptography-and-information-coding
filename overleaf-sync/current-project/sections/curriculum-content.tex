\section{NỘI DUNG CHƯƠNG TRÌNH}

\subsection{Các Khái Niệm Cơ Bản}
\begin{itemize}
\item \textbf{Nguyên tắc Kerckhoff} (Kerckhoff's Principle)
\item \textbf{Tính cứng nhắc tính toán} (Computational Hardness)
\item \textbf{Bảo mật có thể chứng minh} (Provable Security)
\item \textbf{One-time pad} và các khái niệm bảo mật cơ bản
\end{itemize}

\subsection{Các Nguyên Tắc Mật Mã Chính}
\begin{itemize}
\item \textbf{Tạo số ngẫu nhiên giả} (Pseudorandom Number Generation)
\item \textbf{Mã hóa khối} (Block Ciphers)
\item \textbf{Hàm băm} (Hash Functions)
\item \textbf{Chức năng giả ngẫu nhiên} (Pseudorandom Functions)
\item \textbf{Chia sẻ bí mật} (Secret Sharing)
\end{itemize}

\subsection{Mật Mã Đối Xứng}
\begin{itemize}
\item Các phương thức mã hóa đối xứng
\item Tấn công chọn văn bản gốc (Chosen Plaintext Attack)
\item Tấn công chọn văn bản mã hóa (Chosen Ciphertext Attack)
\item \textbf{Mã xác thực tin nhắn} (Message Authentication Codes)
\item \textbf{Mã hóa có xác thực} (Authenticated Encryption)
\end{itemize}

\subsection{Mật Mã Khóa Công Khai}
\begin{itemize}
\item \textbf{Chữ ký số RSA} (RSA Digital Signatures)
\item \textbf{Thỏa thuận khóa Diffie-Hellman} (Diffie-Hellman Key Agreement)
\item Các kỹ thuật mã hóa khóa công khai
\end{itemize}

\subsection{Giao Thức Bảo Mật Thực Tiễn}
\begin{itemize}
\item \textbf{TLS Protocol} (Transport Layer Security)
\item \textbf{Các giao thức nhắn tin bảo mật} (ví dụ: Signal Protocol)
\item \textbf{Mật mã hậu lượng tử} (Post-Quantum Cryptography)
\end{itemize}