\documentclass[12pt,a4paper]{article}
\usepackage{template}

\title{\textbf{CO3083 - MẬT MÃ HỌC VÀ MÃ HÓA THÔNG TIN} \\ 
       \Large Applied Cryptography and Information Coding \\
       \normalsize Tài liệu Tổng quan và Nghiên cứu}
\author{}
\date{}

\begin{document}

\maketitle
\thispagestyle{empty}
\newpage

\tableofcontents
\newpage

\vspace{1cm}
\rule{\textwidth}{1pt}
\vspace{0.5cm}

% Course Overview Section
\section{THÔNG TIN CHUNG}

\subsection{Tên môn học}
CO3083 - Mật mã học và Mã hóa Thông tin (Applied Cryptography and Information Coding)

\subsection{Đối tượng học viên}
\begin{itemize}
\item Sinh viên năm cuối đại học (undergraduate seniors)
\item Sinh viên sau đại học (graduate students)
\item Yêu cầu nền tảng: Có kiến thức toán rời rạc cơ bản
\end{itemize}

\subsection{Ngôn ngữ giảng dạy}
Tiếng Anh

\vspace{0.5cm}
\rule{\textwidth}{1pt}
\vspace{0.5cm}

\section{GIỚI THIỆU VỀ TRƯỜNG ĐẠI HỌC MỸ TẠI BEIRUT (AUB)}

Đại học Mỹ tại Beirut (American University of Beirut - AUB) là một trường đại học tư thục, phi tôn giáo và độc lập, được thành lập từ năm 1866. Trường có vị thế quan trọng trong khu vực Trung Đông và thế giới:

\textbf{Thông số cơ bản:}
\begin{itemize}
\item Xếp hạng QS 2026: Thứ 237 thế giới, thứ 6 thế giới Ả Rập
\item Ngân sách hoạt động: 423 triệu USD
\item Quỹ tài trợ: 768 triệu USD
\item Sinh viên: Hơn 9.000 sinh viên từ 120+ quốc gia
\item Cơ sở vật chất: 64 tòa nhà, 7 ký túc xá, 4 thư viện, 3 bảo tàng
\end{itemize}

\textbf{Lịch sử và Uy tín:}
\begin{itemize}
\item Thành lập: 1862 (mở cửa 1866) với tên ban đầu là Syrian Protestant College
\item Đổi tên thành American University of Beirut: 1920
\item Nhiều cựu sinh viên giữ vị trí lãnh đạo cao trong chính phủ, ngân hàng trung ương và học viện
\end{itemize}

\vspace{0.5cm}
\rule{\textwidth}{1pt}
\vspace{0.5cm}

\section{MỤC TIÊU MÔN HỌC}

Môn học tập trung vào \textbf{lý thuyết và ứng dụng thực tế của mật mã hiện đại}, nhằm:

\begin{itemize}
\item Cung cấp nền tảng lý thuyết vững chắc về mật mã học
\item Phát triển kỹ năng thực hành trong thiết kế và triển khai các giải pháp mật mã
\item Giúp sinh viên có khả năng đánh giá, thiết kế và triển khai các giải pháp mật mã an toàn trong thế giới thực
\item Phát triển kỹ năng trình bày, viết khoa học và nhận thức nghề nghiệp
\end{itemize}

\newpage

% Curriculum Content Section  
\section{NỘI DUNG CHƯƠNG TRÌNH}

\subsection{Các Khái Niệm Cơ Bản}
\begin{itemize}
\item \textbf{Nguyên tắc Kerckhoff} (Kerckhoff's Principle)
\item \textbf{Tính cứng nhắc tính toán} (Computational Hardness)
\item \textbf{Bảo mật có thể chứng minh} (Provable Security)
\item \textbf{One-time pad} và các khái niệm bảo mật cơ bản
\end{itemize}

\subsection{Các Nguyên Tắc Mật Mã Chính}
\begin{itemize}
\item \textbf{Tạo số ngẫu nhiên giả} (Pseudorandom Number Generation)
\item \textbf{Mã hóa khối} (Block Ciphers)
\item \textbf{Hàm băm} (Hash Functions)
\item \textbf{Chức năng giả ngẫu nhiên} (Pseudorandom Functions)
\item \textbf{Chia sẻ bí mật} (Secret Sharing)
\end{itemize}

\subsection{Mật Mã Đối Xứng}
\begin{itemize}
\item Các phương thức mã hóa đối xứng
\item Tấn công chọn văn bản gốc (Chosen Plaintext Attack)
\item Tấn công chọn văn bản mã hóa (Chosen Ciphertext Attack)
\item \textbf{Mã xác thực tin nhắn} (Message Authentication Codes)
\item \textbf{Mã hóa có xác thực} (Authenticated Encryption)
\end{itemize}

\subsection{Mật Mã Khóa Công Khai}
\begin{itemize}
\item \textbf{Chữ ký số RSA} (RSA Digital Signatures)
\item \textbf{Thỏa thuận khóa Diffie-Hellman} (Diffie-Hellman Key Agreement)
\item Các kỹ thuật mã hóa khóa công khai
\end{itemize}

\subsection{Giao Thức Bảo Mật Thực Tiễn}
\begin{itemize}
\item \textbf{TLS Protocol} (Transport Layer Security)
\item \textbf{Các giao thức nhắn tin bảo mật} (ví dụ: Signal Protocol)
\item \textbf{Mật mã hậu lượng tử} (Post-Quantum Cryptography)
\end{itemize}

\vspace{0.5cm}
\rule{\textwidth}{1pt}
\vspace{0.5cm}

% Teaching Methods and Resources Section
\section{PHƯƠNG PHÁP GIẢNG DẠY}

\subsection{Phương pháp chính}
\begin{itemize}
\item \textbf{Game-based approach}: Sử dụng phương pháp "trò chơi" để định nghĩa hình thức và chứng minh bảo mật
\item Tập trung vào \textbf{bảo mật có thể chứng minh} (Provable Security)
\item Kết hợp lý thuyết và thực hành
\end{itemize}

\subsection{"The Key Exchange" - Hoạt động đặc biệt}
\begin{itemize}
\item \textbf{Buổi thảo luận tùy chọn hàng tuần} về các chủ đề mật mã tiên tiến
\item Thảo luận về \textbf{tin tức bảo mật} hiện tại
\item Phát triển kỹ năng \textbf{trình bày và viết khoa học}
\item Tăng cường \textbf{nhận thức nghề nghiệp}
\item Môi trường hỗ trợ cho sinh viên giao lưu và trao đổi kiến thức chuyên sâu
\end{itemize}

\section{TÀI LIỆU HỌC TẬP}

\subsection{Giáo trình chính: ``The Joy of Cryptography''}
\textbf{Tác giả:} Mike Rosulek (Oregon State University)\\
\textbf{Đặc điểm:}
\begin{itemize}
\item Giáo trình đại học \textbf{miễn phí} (Creative Commons BY-NC-SA 4.0)
\item \textbf{286 trang} (bản nháp 2021)
\item Tập trung vào \textbf{bảo mật có thể chứng minh}
\item Sử dụng \textbf{game-based methodology}
\item Được sử dụng tại nhiều trường đại học quốc tế (Oregon State, Johns Hopkins, Boston University)
\end{itemize}

\textbf{Phong cách trình bày:}
\begin{itemize}
\item Thiên về định nghĩa hình thức và chứng minh bảo mật
\item Phương pháp thống nhất và đơn giản hóa
\item Phù hợp với sinh viên có nền tảng toán rời rạc
\end{itemize}

\section{PHƯƠNG PHÁP ĐÁNH GIÁ}

\subsection{Các hình thức đánh giá}
\begin{itemize}
\item \textbf{Bài tập vấn đề} (Problem Sets)
\item \textbf{Dự án phòng thí nghiệm} (Laboratory Projects)
\item \textbf{Bài kiểm tra nhanh} (Quizzes) để củng cố kiến thức
\item \textbf{CryptoHack Challenges} (bắt buộc)
\end{itemize}

\subsection{Yêu cầu CryptoHack}
\begin{itemize}
\item \textbf{Đăng ký username} trên CryptoHack
\item \textbf{Submit link} của username
\item \textbf{Write-up repository} cho các challenges đã giải quyết
\end{itemize}

\section{NGÔN NGỮ LẬP TRÌNH}

\subsection{Ngôn ngữ sử dụng}
\begin{itemize}
\item \textbf{Rust}
\item \textbf{Go}
\end{itemize}

\subsection{Lý do lựa chọn}
\begin{itemize}
\item \textbf{Tính an toàn bộ nhớ} (Memory Safety)
\item \textbf{Hiệu suất cao} (High Performance)
\item \textbf{Thư viện mật mã mạnh} (Strong Cryptographic Libraries)
\end{itemize}

\section{GIÁ TRỊ VÀ ỨNG DỤNG}

\subsection{Kỹ năng đạt được}
\begin{itemize}
\item Hiểu biết sâu sắc về lý thuyết mật mã hiện đại
\item Kỹ năng thiết kế và triển khai các giải pháp mật mã an toàn
\item Khả năng đánh giá và phân tích các hệ thống bảo mật
\item Kỹ năng lập trình an toàn với Rust và Go
\item Kỹ năng trình bày và viết khoa học
\end{itemize}

\subsection{Ứng dụng thực tế}
\begin{itemize}
\item Phát triển các ứng dụng và hệ thống bảo mật
\item Nghiên cứu và phát triển các giao thức mật mã mới
\item Đánh giá và kiểm định bảo mật cho các tổ chức
\item Chuẩn bị cho nghiên cứu sau đại học trong lĩnh vực bảo mật thông tin
\end{itemize}

\newpage

% Vietnam Cryptography Paper Section
\section{TÓM TẮT VÀ DỊCH BÀI BÁO KHOA HỌC}
\subsection{"CRYPTOGRAPHY DURING THE FRENCH AND AMERICAN WARS IN VIETNAM"}
\subsubsection{(Mật mã học trong các cuộc chiến tranh Pháp và Mỹ tại Việt Nam)}

\vspace{0.5cm}
\rule{\textwidth}{1pt}
\vspace{0.5cm}

\subsection{THÔNG TIN CƠ BẢN}

\textbf{Tiêu đề gốc:} Cryptography During the French and American Wars in Vietnam\\
\textbf{Tiêu đề dịch:} Mật mã học trong các cuộc chiến tranh Pháp và Mỹ tại Việt Nam

\textbf{Tác giả:}
\begin{itemize}
\item \textbf{Phan Dương Hiệu} - Giáo sư tại XLIM, Đại học Limoges, Pháp (Tiến sĩ mật mã học từ École Normale Supérieure năm 2005)
\item \textbf{Neal Koblitz} - Giáo sư tại Đại học Washington (Tiến sĩ toán học từ Princeton năm 1974)
\end{itemize}

\textbf{Xuất bản:}
\begin{itemize}
\item IACR Cryptology ePrint Archive, Paper 2016/1136 (2016)
\item Cryptologia, Volume 41, Issue 6, pp. 491-511 (2017)
\end{itemize}

\textbf{Từ khóa:} Chiến tranh Việt Nam, tình báo thông tin, bảo mật thông tin liên lạc

\vspace{0.5cm}
\rule{\textwidth}{1pt}
\vspace{0.5cm}

\subsection{TÓM TẮT CHÍNH (ABSTRACT - DỊCH)}

Sau Tuyên ngôn Độc lập của Việt Nam ngày 2 tháng 9 năm 1945, đất nước phải trải qua hai cuộc chiến tranh dài và tàn khốc, đầu tiên chống lại Pháp sau đó chống lại Mỹ, trước khi cuối cùng vào năm 1975 trở thành một đất nước thống nhất thoát khỏi ách thống trị thực dân. Mục đích của nghiên cứu này là xem xét vai trò của mật mã học trong hai cuộc chiến tranh đó. Mặc dù đối thủ có tài nguyên công nghệ lớn hơn nhiều, các chuyên gia tình báo thông tin của Việt Minh, Mặt trận Dân tộc Giải phóng miền Nam Việt Nam và Cộng hòa Dân chủ Việt Nam đã có thành công đáng kể trong việc bảo vệ thông tin liên lạc của Việt Nam và thu thập các bí mật chiến thuật, chiến lược từ kẻ thù. Có lẽ đáng ngạc nhiên, trong cả hai cuộc chiến tranh đều có sự cân bằng giữa các bên. Nói chung, kiến thức mật mã và thiết kế giao thức ở mức cao tại các bộ chỉ huy trung ương, nhưng việc triển khai cho thông tin liên lạc chiến thuật trên thực địa gặp khó khăn, và có nhiều thất bại ở tất cả các bên.

\vspace{0.5cm}
\rule{\textwidth}{1pt}
\vspace{0.5cm}

\subsection{NỘI DUNG CHÍNH CỦA BÀI NGHIÊN CỨU}

\subsubsection{Bối cảnh lịch sử và ý nghĩa}

Những chiến thắng - gây sốc và bất ngờ đối với nhiều người phương Tây - của một dân tộc lạc hậu về công nghệ trước hai cường quốc công nghiệp phương Tây tiên tiến là những sự kiện đặc biệt của thế kỷ 20. Bài báo này có ý nghĩa quan trọng vì:

\begin{itemize}
\item Đây là nghiên cứu khoa học đầu tiên xem xét vai trò của mật mã học trong các cuộc chiến tranh Việt Nam
\item Cung cấp góc nhìn về cách một quốc gia có công nghệ lạc hậu có thể thành công trong lĩnh vực tình báo thông tin
\item Khuyến khích các nước đang phát triển ngày nay tự tin thoát khỏi sự phụ thuộc vào kiến thức và sản phẩm nhập khẩu
\end{itemize}

\subsubsection{Cuộc chiến chống Pháp (1945-1954) và thời kỳ trung gian (1954-1960)}

\textbf{Những năm đầu (1945-1946)}

Ngay từ đầu, lãnh đạo tại Hà Nội đã coi trọng tình báo thông tin. Theo lịch sử của chính phủ Việt Nam được NSA dịch (NSA 2014), chi nhánh mật mã của Lực lượng Vũ trang Nhân dân được thành lập ngày 12 tháng 9 năm 1945, chỉ mười ngày sau Tuyên ngôn Độc lập của Việt Nam.

\textbf{Trình độ mật mã ban đầu:}
\begin{itemize}
\item Vào thời điểm đó, trình độ mật mã của người Việt Nam chưa cao. Như được mô tả trong lịch sử của Cục Mật mã (Ban Cơ Yếu n.d.), hệ thống họ đang sử dụng vào cuối năm 1945 và đầu năm 1946 chỉ ít hơn một chút so với mật mã Caesar.
\item Sử dụng các phương pháp mật mã cơ bản và thủ công
\item Dần dần phát triển và cải thiện hệ thống theo thời gian
\end{itemize}

\textbf{Điện Biên Phủ và thành công của Việt Minh}

Chiến thắng tại Điện Biên Phủ vào mùa xuân năm 1954 đánh dấu sự nhục nhã của Pháp và là khởi đầu của sự kết thúc chủ nghĩa thực dân Pháp; đó là nguồn cảm hứng cho những người khác, chủ yếu ở Bắc Phi, những người đang chịu đựng dưới ách thống trị thực dân Pháp và đã giành được độc lập vài năm sau đó.

\subsubsection{Cuộc chiến chống Mỹ (1960-1975)}

\textbf{Sự phát triển của hệ thống mật mã Việt Nam}

\textbf{Thành tựu đáng chú ý:}
\begin{itemize}
\item Phát triển các phương pháp mật mã tiên tiến hơn so với thời kỳ chống Pháp
\item Xây dựng mạng lưới tình báo thông tin hiệu quả
\item Thành công trong việc bảo vệ thông tin liên lạc trước công nghệ tiên tiến của Mỹ
\end{itemize}

\textbf{Chiến thắng cuối cùng}

Việc trục xuất lực lượng Mỹ khỏi miền Nam Việt Nam vào ngày 30 tháng 4 năm 1975 — đây là lần duy nhất Hoa Kỳ từng bị đánh bại một cách quyết định trong một cuộc chiến tranh — đã mang lại sự khích lệ to lớn cho những người khác, đặc biệt ở Mỹ Latinh, những người đang đấu tranh chống lại Hoa Kỳ.

\subsubsection{Phân tích kỹ thuật và phương pháp}

\textbf{Đặc điểm chung của mật mã trong hai cuộc chiến}

\textbf{Ở cấp độ chỉ huy trung ương:}
\begin{itemize}
\item Kiến thức mật mã và thiết kế giao thức ở mức cao
\item Có hệ thống mật mã phức tạp và hiệu quả
\item Quản lý thông tin tình báo chặt chẽ
\end{itemize}

\textbf{Ở cấp độ chiến thuật (thực địa):}
\begin{itemize}
\item Triển khai gặp nhiều khó khăn
\item Có nhiều thất bại ở tất cả các bên
\item Thử thách về mặt kỹ thuật và logistics
\end{itemize}

\textbf{Sự cân bằng bất ngờ}

Mặc dù có sự chênh lệch lớn về công nghệ và tài nguyên, nhưng trong cả hai cuộc chiến tranh đều có \textbf{sự cân bằng tương đối} giữa các bên về mặt mật mã học và tình báo thông tin.

\subsubsection{Các cơ quan và tổ chức liên quan}

\textbf{Bên Việt Nam:}
\begin{itemize}
\item \textbf{Việt Minh} (1945-1954)
\item \textbf{Mặt trận Dân tộc Giải phóng miền Nam Việt Nam} (NLF)
\item \textbf{Cộng hòa Dân chủ Việt Nam} (DRV)
\item \textbf{Cục Mật mã} và các cơ quan tình báo thông tin
\end{itemize}

\textbf{Bên đối phương:}
\begin{itemize}
\item Các lực lượng thực dân Pháp và hệ thống tình báo
\item Quân đội Mỹ và các cơ quan tình báo Mỹ (NSA, CIA)
\end{itemize}

\subsubsection{Tài liệu và nguồn tham khảo}

Bài báo dựa trên:
\begin{itemize}
\item Tài liệu lịch sử của chính phủ Việt Nam được NSA dịch (NSA 2014)
\item Lịch sử của Cục Mật mã (Ban Cơ Yếu n.d.)
\item Các tài liệu declassified từ các cơ quan tình báo
\item Phỏng vấn và hồi ký của các nhân chứng
\end{itemize}

\vspace{0.5cm}
\rule{\textwidth}{1pt}
\vspace{0.5cm}

\subsection{Ý NGHĨA VÀ ĐÓNG GÓP KHOA HỌC}

\subsubsection{Đóng góp cho lịch sử mật mã học}
\begin{itemize}
\item Đây là nghiên cứu đầu tiên xem xét một cách hệ thống vai trò của mật mã học trong các cuộc chiến tranh Việt Nam
\item Cung cấp góc nhìn mới về lịch sử mật mã học từ phía một quốc gia đang phát triển
\item Làm phong phú thêm tài liệu nghiên cứu về lịch sử mật mã học thế giới
\end{itemize}

\subsubsection{Bài học về bảo mật thông tin}
\begin{itemize}
\item Chứng minh rằng kiến thức và kỹ năng có thể bù đắp cho sự thiếu hụt về công nghệ
\item Tầm quan trọng của việc bảo vệ thông tin liên lạc trong chiến tranh
\item Vai trò của tình báo thông tin trong chiến lược quân sự
\end{itemize}

\subsubsection{Giá trị giáo dục và cảm hứng}
\begin{itemize}
\item Nhận thức về lịch sử này có thể mang lại cho người dân các nước đang phát triển ngày nay sự tự tin cần thiết để thoát khỏi sự phụ thuộc gần như hoàn toàn vào kiến thức nhập khẩu và sản phẩm nhập khẩu.
\item Khuyến khích việc phát triển công nghệ và kiến thức bản địa
\end{itemize}

\vspace{0.5cm}
\rule{\textwidth}{1pt}
\vspace{0.5cm}

\subsection{KẾT LUẬN}

Bài báo "Cryptography During the French and American Wars in Vietnam" của Phan Dương Hiệu và Neal Koblitz là một nghiên cứu lịch sử quan trọng và độc đáo về vai trò của mật mã học trong các cuộc chiến tranh giải phóng dân tộc của Việt Nam.

\textbf{Những điểm nổi bật:}

\begin{enumerate}
\item \textbf{Tính khoa học cao:} Được xuất bản trên các tạp chí uy tín quốc tế về mật mã học
\item \textbf{Góc nhìn mới:} Xem xét lịch sử mật mã học từ phía một quốc gia đang phát triển
\item \textbf{Ý nghĩa lịch sử:} Làm sáng tỏ vai trò quan trọng của mật mã học trong chiến thắng của Việt Nam
\item \textbf{Giá trị giáo dục:} Cung cấp bài học về sự kiên trì, sáng tạo và khả năng thích ứng
\item \textbf{Cảm hứng cho hiện tại:} Khuyến khích các nước đang phát triển tự chủ về công nghệ
\end{enumerate}

Bài báo này không chỉ có giá trị về mặt lịch sử mà còn mang ý nghĩa thực tiễn cho việc phát triển bảo mật thông tin và mật mã học hiện đại, đặc biệt là ở các quốc gia đang phát triển.

\vspace{1cm}
\rule{\textwidth}{1pt}
\vspace{0.5cm}

\begin{center}
\textit{Lưu ý: Đây là bản tóm tắt dựa trên thông tin có sẵn từ abstract và các nguồn tài liệu công khai. Để hiểu đầy đủ nội dung chi tiết, cần truy cập toàn văn bài báo gốc.}
\end{center}

\vspace{1cm}
\rule{\textwidth}{1pt}
\vspace{0.5cm}

\section{KẾT LUẬN CHUNG}

Môn học CO3083 tại AUB cung cấp một chương trình toàn diện về mật mã học hiện đại, kết hợp giữa nền tảng lý thuyết vững chắc và các kinh nghiệm thực hành. Với sự hỗ trợ từ giáo trình chất lượng cao, các hoạt động thảo luận chuyên sâu và yêu cầu thực hành qua CryptoHack, sinh viên sẽ được trang bị đầy đủ kiến thức và kỹ năng để thành công trong lĩnh vực bảo mật thông tin và mật mã học.

Việc nghiên cứu lịch sử mật mã học Việt Nam qua bài báo của Phan Dương Hiệu và Neal Koblitz không chỉ mang giá trị học thuật mà còn cung cấp cảm hứng và bài học quý báu về sự kiên trì, sáng tạo trong điều kiện hạn chế về tài nguyên.

\vspace{1cm}
\rule{\textwidth}{1pt}
\vspace{0.5cm}

\begin{center}
\textit{Tài liệu được tổng hợp từ thông tin khóa học Applied Cryptography tại American University of Beirut \\
và nghiên cứu lịch sử mật mã học Việt Nam}
\end{center}

\end{document}