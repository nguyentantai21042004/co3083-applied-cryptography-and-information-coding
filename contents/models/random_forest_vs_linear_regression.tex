

\section{So sánh mô hình Random Forest - Linear Regression}

\subsection{Quy trình xử lý dữ liệu}

\subsubsection{Điểm tương đồng}
\begin{itemize}
  \item Cả hai mô hình đều tiến hành xử lý dữ liệu thiếu và chuyển đổi định dạng thời gian
  \item Cả hai đều sử dụng chuẩn hóa dữ liệu bằng \texttt{StandardScaler}
  \item Cả hai đều chia dữ liệu theo tỷ lệ 80\% huấn luyện - 20\% kiểm thử
\end{itemize}

\subsubsection{Điểm khác biệt}

\begin{table}[H]
\centering
\caption{So sánh quy trình xử lý dữ liệu}
\begin{tabular}{p{7cm}|p{7cm}}
\toprule
\textbf{Random Forest} & \textbf{Linear Regression} \\
\midrule
Tập trung nhiều vào feature engineering với nhiều đặc trưng phức tạp: tương tác giữa các đặc trưng, đặc trưng đa thức, đặc trưng thời gian dạng sine/cosine, tỷ lệ giữa các đặc trưng & Sử dụng phương pháp \texttt{KNNImputer} để xử lý giá trị thiếu \\
\midrule
Đặc biệt sử dụng các đặc trưng cửa sổ trượt (rolling features) & Áp dụng phương pháp IQR để loại bỏ outliers \\
\midrule
 & Feature engineering đơn giản hơn, chỉ thêm 3 đặc trưng phi tuyến \\
\midrule
 & Ràng buộc hệ số không âm cho mô hình (\texttt{positive=True}) \\
\bottomrule
\end{tabular}
\end{table}

\subsection{Đặc trưng quan trọng và hệ số}

\begin{table}[H]
\centering
\caption{So sánh đặc trưng quan trọng giữa hai mô hình}
\begin{tabular}{>{\raggedright\arraybackslash}p{7cm}|>{\raggedright\arraybackslash}p{7cm}}
\toprule
\textbf{Random Forest} & \textbf{Linear Regression} \\
\midrule
Đặc trưng quan trọng nhất: \texttt{NOx(GT)\_rolling\_std} ($\sim$0.6) và \texttt{NO2(GT)\_rolling\_mean} ($\sim$0.37) & Đặc trưng quan trọng nhất: \texttt{C6H6(GT)} (hệ số 0.577) \\
\midrule
Các đặc trưng cửa sổ trượt đóng vai trò quan trọng nhất & Tiếp theo là \texttt{PT08.S1(CO)} (0.145), \texttt{PT08.S3(NOx)} (0.127) ,  \texttt{NOx(GT)} (0.126) \\
\midrule
Phần lớn các đặc trưng khác có tầm quan trọng thấp & Các đặc trưng được tạo thêm như \texttt{NO2\_GT*RH} có tầm ảnh hưởng trung bình \\
\bottomrule
\end{tabular}
\end{table}

\subsubsection{Phân tích ma trận tương quan}
\begin{itemize}
  \item Từ biểu đồ ma trận tương quan (Hình 2), ta thấy \texttt{C6H6(GT)} có tương quan cao với \texttt{CO(GT)}
  \item Có tương quan âm mạnh giữa \texttt{PT08.S3(NOx)} với nhiều đặc trưng khác
  \item Dữ liệu thời tiết (T, RH, AH) có tương quan mạnh với nhau
\end{itemize}

\subsection{Hiệu suất mô hình}

\begin{table}[H]
\centering
\caption{So sánh hiệu suất giữa hai mô hình}
\begin{tabular}{l|c|c}
\toprule
\textbf{Chỉ số} & \textbf{Linear Regression} & \textbf{Random Forest} \\
\midrule
$R^2$ & 0.8673 & 0.9992 \\
\midrule
RMSE & 0.3118 & 0.0136 \\
\midrule
MAE & 0.2088 & 0.0053 \\
\midrule
MSE & 0.0972 &  0.0002\\
\bottomrule
\end{tabular}
\end{table}

\subsection{Phân tích biểu đồ dự đoán}

\begin{table}[H]
\centering
\caption{So sánh kết quả biểu đồ dự đoán}
\begin{tabular}{p{7cm}|p{7cm}}
\toprule
\textbf{Random Forest} & \textbf{Linear Regression} \\
\midrule
Có xu hướng dự đoán kém ở giá trị CO cao ($>$200) & Biểu đồ so sánh giá trị thực tế và dự đoán (Hình 1) cho thấy sự tuyến tính khá rõ ràng \\
\midrule
Phần dư tăng theo giá trị thực tế, thiên lệch ở giá trị cao & Biểu đồ phân tán sai số (Hình 4) cho thấy phần dư phân bố khá đồng đều quanh 0 \\
\midrule
Biểu đồ phần dư cho thấy nhiều điểm ngoại lai có phần dư lớn ($\sim$150-170) & Phân phối sai số (Hình 5) có dạng chuông, tập trung quanh 0, cho thấy mô hình có độ chính xác cao \\
\bottomrule
\end{tabular}
\end{table}

\subsection{Ưu điểm và nhược điểm}

\begin{table}[h]
\centering
\caption{So sánh ưu điểm của hai mô hình}
\begin{tabular}{p{7cm}|p{7cm}}
\toprule
\textbf{Random Forest} & \textbf{Linear Regression} \\
\midrule
Khả năng nắm bắt mối quan hệ phi tuyến phức tạp không cần xác định trước & Đơn giản, dễ diễn giải thông qua các hệ số \\
\midrule
Tận dụng hiệu quả các đặc trưng cửa sổ trượt để nắm bắt xu hướng thời gian & Hiệu suất rất tốt ($R^2 = 0.87$) với mô hình đơn giản \\
\midrule
Ít bị ảnh hưởng bởi outliers do tính chất ensemble & Phân phối sai số chuẩn, tập trung quanh 0 \\
\midrule
 & Xử lý outliers hiệu quả bằng IQR \\
\bottomrule
\end{tabular}
\end{table}

\begin{table}[h]
\centering
\caption{So sánh nhược điểm của hai mô hình}
\begin{tabular}{p{7cm}|p{7cm}}
\toprule
\textbf{Random Forest} & \textbf{Linear Regression} \\
\midrule
Không dự đoán tốt các giá trị CO cao (extreme values) & Giả định tuyến tính có thể không phù hợp với một số mối quan hệ phức tạp \\
\midrule
Mô hình phức tạp hơn, khó diễn giải hơn & Cần ràng buộc hệ số không âm để phù hợp với bài toán vật lý \\
\midrule
Cần nhiều đặc trưng kỹ thuật hơn để hoạt động tốt & Có thể bị ảnh hưởng bởi các đặc trưng có đa cộng tuyến cao \\
\bottomrule
\end{tabular}
\end{table}

\subsection{Kết luận và đề xuất}


\begin{table}[H]
\centering
\caption{Tổng kết so sánh hai mô hình}
\begin{tabular}{p{3cm}|p{5.5cm}|p{5.5cm}}
\toprule
\textbf{Tiêu chí} & \textbf{Random Forest} & \textbf{Linear Regression} \\
\midrule
Hiệu suất & Gặp khó khăn với các giá trị CO cao & Hiệu suất rất khả quan ($R^2 = 0.87$) \\
\midrule
\raggedright Đặc trưng quan trọng & Coi trọng đặc trưng cửa sổ trượt (rolling features) & Nhấn mạnh vai trò của \texttt{C6H6(GT)} \\
\midrule
Độ phức tạp & Phức tạp hơn nhiều về cả feature engineering và cấu trúc mô hình & Đơn giản và dễ hiểu hơn \\
\midrule
Phân phối sai số & Có nhiều điểm ngoại lai với sai số lớn & Phân phối sai số đẹp hơn, tập trung quanh 0 \\
\bottomrule
\end{tabular}
\end{table}

